\documentclass{article}
\usepackage{algorithm}
\usepackage{algorithmic}
\usepackage[spanish]{babel}

\begin{document}
\title{Agoritmos en \LaTeX}
\author{Nauman \\ recluze@gmail.com}
\maketitle



\begin{algorithm}
\begin{algorithmic}
\REQUIRE{entradas de algo}
\ENSURE{salidas generadas}

\STATE x=2;
\IF{i $\to$ 2}
   \STATE y--;
   \STATE x++;		\label{line:impline}
\ELSE
   \RETURN x		\label{line:mostimp}
\ENDIF
\end{algorithmic}
\caption{Mi Primer Algoritmo Simple}
\label{algo:first}
\end{algorithm}

Y, por supuesto, podemos referirnos al algoritmo usando \verb|\ref|: Ver Algoritmo~\ref{algo:first} pero lo bueno es que también podemos referirnos a una línea específica, p. Línea~\ref{line:impline} o Línea~\ref{line:mostimp}.

\end{document}