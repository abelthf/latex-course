\documentclass{article}
\usepackage{amsmath}       % new


\begin{document}
\title{Matemáticas Tipográficas en \LaTeX}
\author{Abel Huanca \\ abel.huanca@upeu.edu.pe}
\maketitle


\section{Introducción} 
\LaTeX\ es extremadamente poderoso cuando se trata de composición matemática. Es uno de los puntos fuertes de este sistema. 

\section{Viendo matemáticas}
Hay dos formas de mostrar las matemáticas. Uno es  \emph{inline} y el otro es formato \emph{display} format -- en el que toda la matemática se encuentra en su propio conjunto de líneas.


\subsection{Modo Inline}
Vamos a insertar una ecuación matemática en línea aquí usando un par de signos \$ signs:    . AsComo puede ver, la pantalla (como el espaciado de línea) no se ve afectada por las matemáticas como lo hace con los softwares de procesamiento de texto.

\subsection{Modo Display}
También podemos mostrar ecuaciones en su propio conjunto de líneas. Para hacer esto, podemos usar el entorno de ecuaciones.

%\begin{equation}\label{eq:emc}
%
%\end{equation}

Como puedes ver, \LaTeX\ inserta el número de ecuación automáticamente. Podemos referirnos a él usando el \verb|\ref| comando tal como nos referimos a secciones, figuras y tablas. (Por ejemplo, ecuación~\ref{eq:emc}.) Para deshacerse del número de ecuación, simplemente use  \emph{star variant} del entorno de la ecuación. (Para esto, necesitas el paquete \texttt{amsmath}.)

%\begin{equation*}
%
%\end{equation*}

Alternativamente, podemos usar las teclas abreviadas \verb|\[| y \verb|\]|



\section{Características Matemáticas}
\LaTeX\ tiene muchas características integradas y puede obtener muchas más fácilmente. Aquí, veremos algunas de estas características:

Suma, resta, multiplicación y división:


Superíndices y subíndices:



Suma, unión, intersección, unión grande, integrales:



Fracciones, corchetes, raíz cuadrada:


Letras griegas:



Matrices y vectores. Para esto, debe incluir el paquete \texttt {amsmath} y luego usar el entorno \texttt {bmatrix} o \texttt {pmatrix}:


Acentos:



Vea el menú \texttt {Math} en el IDE para otras operaciones. Puede consultar la `` Guía breve de matemáticas para \LaTeX '' para obtener muchos más ejemplos. 

\section{Usando Símbolos} 
Es posible que encuentre situaciones en las que necesite encontrar nuevos símbolos. Para esto, puede consultar la `` Lista completa de símbolos \LaTeX ''.






(Opcional) Dado que este es un comando largo, es posible que deseemos crear un acceso directo usando \verb| \newcommand | comando en el preámbulo. Esto también nos permite cambiar más tarde el símbolo sin tener que cambiar las ecuaciones.


\end{document}